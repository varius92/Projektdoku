% !TEX root = ../Projektdokumentation.tex
\section{Projektabschluss} 
\label{sec:Projektabschluss}

\subsection{Abnahme} 
\label{sec:Abnahme}
Während der Abnahmephase wurde das Projekt in die Zielumgebung eingeführt und
nochmals nach Darstellungsfehler untersucht. Währenddessen füllt das
Projektmanagement die Website mit dem endgültigen Content. Zusätzlich wurde der
Mitarbeiter des Projektmanagments auf Besonderheiten und Backendabläufe
hingewiesen. Diese gibt er dem Kunden im Form einer Schulung kund.

\subsection{Soll-/Ist-Vergleich}
\label{sec:SollIstVergleich}

Zurückschauend ist zu bemerken, dass 3 Stunden mehr gebraucht wurde,
als im zuvor unter dem Kapitel \nameref{sec:Projektphasen} festgelegten Zeitplan
ursprünglich geplant war.
Die Gründe waren hauptsächlich Änderungsvorgaben des Kunden und Missverständnise
bei der Kommunikation mit den Mitarbeiter der anderen Abteilungen. 

\tabelle{Soll-/Ist-Vergleich}{tab:Vergleich}{Zeitnachher.tex}

Da alle Anforderungen aus dem Lastenheft implementiert und erfüllt worden sind,
hat der Kunde seine Zufriedenheit kund gegeben.


\subsection{Fazit} 
\label{sec:Fazit}

In diesem Projekt habe ich zum ersten mal die \ac{BEM} Methodik angewendet und
konnte mich durch die Vorteile, die die Methodik mit sich bringt, überzeugen
lassen es bei weiteren Projekten ebenso einzusetzen.

Für mich persöhnlich hat sich bei diesem Projekt wieder mal bemerkbar gemacht
wie man im Verlauf der Ausbildung bei wiederholenden Aufgaben stets sicherer und genauer wird. Zu dem
kommt hinzu, dass man bei Abschluss eines Web Projektes ein für alle sichbares
und verwendbares Ergebnis hat, was einen mit einem guten Gefühl belohnt.

