% !TEX root = ../Projektdokumentation.tex
\section{Projektabschluss} 
\label{sec:Projektabschluss}

\subsection{Abnahme} 
\label{sec:Abnahme}
Während der Abnahmephase wurde das Projekt in die Zielumgebung eingeführt und
nochmals nach Darstellungsfehler untersucht. Währenddessen füllt das
Projektmanagement die Website mit dem endgültigen Content. Zusätzlich wurde der
Mitarbeiter des Projektmanagments auf Besonderheiten und Backendabläufe
hingewiesen. Diese gibt er dem Kunden im Form einer Schulung kund.

\subsection{Soll-/Ist-Vergleich}
\label{sec:SollIstVergleich}

Zurückschauend, ist es es zu bemerken, dass etwas mehr Zeit gebraucht wurde,
als im zuvor unter \xx Kapitel festgelegten Zeitplan zu sehen ist. Dennoch
zeigte der Ku
\begin{itemize}
	\item Wurde das Projektziel erreicht und wenn nein, warum nicht?
	\item Ist der Auftraggeber mit dem Projektergebnis zufrieden und wenn nein, warum nicht?
	\item Wurde die Projektplanung (Zeit, Kosten, Personal, Sachmittel) eingehalten oder haben sich Abweichungen ergeben und wenn ja, warum?
	\item Hinweis: Die Projektplanung muss nicht strikt eingehalten werden. Vielmehr sind Abweichungen sogar als normal anzusehen. Sie müssen nur vernünftig begründet werden (\zB durch Änderungen an den Anforderungen, unter-/überschätzter Aufwand).
\end{itemize}

\paragraph{Beispiel (verkürzt)}
Wie in Tabelle~\ref{tab:Vergleich} zu erkennen ist, konnte die Zeitplanung bis auf wenige Ausnahmen eingehalten werden.
\tabelle{Soll-/Ist-Vergleich}{tab:Vergleich}{Zeitnachher.tex}

\subsection{Dokumentation}
\label{sec:Dokumentation}

\begin{itemize}
	\item Wie wurde die Anwendung für die Benutzer/Administratoren/Entwickler dokumentiert (\zB Benutzerhandbuch, \acs{API}-Dokumentation)?
	\item Hinweis: Je nach Zielgruppe gelten bestimmte Anforderungen für die Dokumentation (\zB keine IT-Fachbegriffe in einer Anwenderdokumentation verwenden, aber auf jeden Fall in einer Dokumentation für den IT-Bereich).
\end{itemize}

\paragraph{Beispiel}
Ein Ausschnitt aus der erstellten Benutzerdokumentation befindet sich im \Anhang{app:BenutzerDoku}.
Die Entwicklerdokumentation wurde mittels PHPDoc\footnote{Vgl. \cite{phpDoc}} automatisch generiert. Ein beispielhafter Auszug aus der Dokumentation einer Klasse findet sich im \Anhang{app:Doc}. 

\Zwischenstand{Dokumentation}{Dokumentation}



\subsection{Fazit} 
\label{sec:Fazit}






\begin{itemize}
	\item Was hat der Prüfling bei der Durchführung des Projekts gelernt (\zB Zeitplanung, Vorteile der eingesetzten Frameworks, Änderungen der Anforderungen)?
\end{itemize}



\begin{itemize}
	\item Wie wird sich das Projekt in Zukunft weiterentwickeln (\zB geplante Erweiterungen)?
\end{itemize}



