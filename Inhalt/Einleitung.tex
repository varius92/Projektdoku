% !TEX root = ../Projektdokumentation.tex
\section{Einleitung}
\label{sec:Einleitung}
TEXT\ldots 

\subsection{Projektumfeld} 
\label{sec:Projektumfeld}
Die Firma \mh wurde von der Touristeninformation \kunde beauftragt, deren
unzeitgemäßen Internetauftritt mithilfe eines moderneren Erscheinungsbildes in
Form eines Relaunch zu erneuern. Dazu wurde das Screendesign von einer
externen Werbeagentur bereitgestellt.

Die \mh mit Sitz in Bad Reichenhall ist eine crossmedia Werbeagentur, 
die seit 2001 besteht und mittlerweile über 30 Mitarbeiter beschäftigt. 
Die Agentur unterteilt sich in die vier
Abteilungen: Projektmanagement, Kreation- und Design, Web-Technologie \& Design
sowie IT Services. Unter anderem durch die geografische Lage des Unternehmens 
begründet hat die Agentur ihren Schwerpunkt im Bereich Destinationstourismus.
Dementsprechend sind die meisten Softwarelösungen, die die \mh anbietet, darauf ausgelegt.
Zum Angebotsspekturm des Unternehmens gehört neben dem \ac{CMS} \ct 
auch Softwarelösungen wie digiMELDE als Online-Meldewesen Plattform, makroCARD
für Customer Loyality Management und ein innovatives Werkzeug zur
Erstellung von Gastgeberverzeichnissen -- digiAIR (\Fachbegriff{Web2Print}).


%Das \acs{CMS} \ct wird bereits von vielen großen und kleinen
%Destinationen bei der Pflege derer Webauftritten erfolgreich eingesetzt, wie
% \zB von Berchtesgadener Land Tourismus, Predigtstuhlbahn oder Alpine Welten – Der Bergführer.

\subsection{Projektziel} 
\label{sec:Projektziel}

Das Bestreben des Auftraggebers ist es, durch die neue Website dem Besucher
relevante Informationen zu bieten, Emotionen zu vermitteln, das Erfüllen
moderner Anforderungen sowie Buchungen und Anfragen zu generieren.

Ein Hauptaugenmerk ist hierbei, eine optimale und geräteunabhängige Darstellung
im Web zu verwirklichen -- \ac{RWD}. Beim \ac{RWD} erfolgt der
grafische Aufbau der Website auf Basis der Anforderung des jeweiligen Endgerätes
auf dem die Website geöffnet wird. Die Umstellung auf \ac{RWD} führt demnach
zur einer vereinfachten und optimierten Bedienung in allen gängigen
Darstellungsgrößen. Daraus erschließt sich das Ziel eine bestmögliche
Benutzererfahrung zu schaffen.

Ein weiteres Anliegen ist die zeitaufwändige und komplizierte Einpflege von
Inhalten und weitere technische Mängel wie \zB dürftige \ac{SEO} mit einem
\ac{CMS} zu umgehen.


\subsection{Projektbegründung} 
\label{sec:Projektbegruendung}
Basierend auf der Kernzielgruppe "`50+"' ist eine übersichtliche und
anpassungsfähige Darstellung von relevanten Informationen signifikant. 
Die statische Darbietung auf verschiedenen Displaygrößen benachteiligt die
\Fachbegriff{Userability} schwerwiegend und führt dazu, dass
überdurchschnittlich viele Nutzer die Website verlassen. Durch das modernere
Erscheinungsbild in Kombination mit \ac{RWD} kann dies verhindert werden.
 
Ebenso unvorteilhaft ist die lückenhafte Suchmaschinenoptimierung, die die Suche
der Website erschwert. Hinzu kommen noch Performance- und
Sicherheitsproblemen, die durch die veraltete Technik der aktuellen Website
ausgelöst werden. Die eben gennanten Hindernissen als auch die komplizierte und
zeitaufwändige Einpflege von Inhalten kann durch die Implementierung des heuseigenen
sowie modernen \ac{CMS} \ct umgangen werden.

Auf Grund dieser Lösungsansätzen und die dafür nötigen Kompetenzen hat sich die
\kunde entschlossen, der Firma \mh die Entwicklung einer neuen Website in
Auftrag zu geben.

\subsection{Projektschnittstellen} 
\label{sec:Projektschnittstellen}



Als Grundlage für die Umsetzung der Website liegen mir vor:
\begin{itemize}
	\item Mehrere Screendesigns der Start- und Unterseiten in verschiedenen
Darstellungsgrößen (Desktop, Tablet, Mobil) als Bilddateien
	\item Styletemplate mit Informationen zu Schriftgrößen, Schriftstärken,
Elementgrößen, Hover-Effekte, Animationen, \usw vom Screendesigner
	\item Grobkonzept der Navigation vom Projektmanagement
	\item SVG Logo und Icons vom Screendesigner
	\item Zugeschnittene Bilder (\Fachbegriff{Slices}) und Partnerlogos als
	Bilddateien vom Screendesigner
\end{itemize}

------

\begin{itemize}
	\item Mit welchen anderen Systemen interagiert die Anwendung (technische Schnittstellen)?
	\item Wer genehmigt das Projekt \bzw stellt Mittel zur Verfügung? 
	\item Wer sind die Benutzer der Anwendung?
	\item Wem muss das Ergebnis präsentiert werden?
\end{itemize}
zB. externe Werbeagentur?







%-------------------------------------------------------------------------
%-------------------------------------------------------------------------
%-------------------------------------------------------------------------
-------------------------------------------------------------------------




Die Implementierung der Website geschieht in 3 Schritten: \newline
Die Entwicklung eines HTML Dummies unter Verwendung von \acs{HTML}5 und
\acs{CSS}3 \bzw  \acs{LESS} nach \acs{BEM} Methodik. Folgend die Ausarbeitung
der JavaScript \bzw jQuery Funktionalität im Frontend. Anschließend die Integration in 
das firmeneigenen Content Management System \ct sowie dessen Anpassung.






\DesignPattern{Block-Element-Modifier} kurz \acs{BEM} ist eine moderne Methode
der Frontend-Architektur die erlaubt, Webprojekte modularer zu entwickeln. Das erleichtert die Wartbarkeit und macht es
Entwicklern einfacher, die Codebasis jederzeit zu erweitern und trotzdem die Übersicht zu
behalten. \textcolor{red}{Verweis!!!!!!}

Die Anbindung an das firmeneigenen Content Management System \ct
stellt dem Auftraggeber eine unkomplizierte Contentpflege ohne Programmierkenntnisse in einem
responsiven Backend zur Verfügung. Das \acs{CMS} wird nach Anforderungen des
Auftraggebers mithilfe des Musters \DesignPattern{Model-View-Controller} \acs{MVC} erweitert
als auch angepasst.

Für die Umsetzung stehen mir mein Arbeitsrechner (PHPStorm als \acs{IDE}),
sowie die gesamte firmennetz-interne Produktivumgebung (\zB \acs{GIT}, phpMyAdmin) zur Verfügung.

Aufgrund dessen, dass die \acs{BEM}-Methodik zum ersten mal im Unternehmen
verwendet wird, kann dieser Auftrag auch als Leuchtturmprojekt für die \mh 
dienen.
%-------------------------------------------------------------------------
%-------------------------------------------------------------------------
%-------------------------------------------------------------------------
-------------------------------------------------------------------------