% !TEX root = ../Projektdokumentation.tex
\section{Einleitung}
\label{sec:Einleitung}
TEXT\ldots LaTeX

\subsection{Projektumfeld} 
\label{sec:Projektumfeld}
Die Firma \mh wurde von der Touristeninformation \kunde beauftragt, deren
unzeitgemäßen Internetauftritt mithilfe eines moderneren Erscheinungsbildes in
Form eines Relaunch zu erneuern. Dazu wurde das Screendesign von einer
externen Werbeagentur bereitgestellt.

Die \mh mit Sitz in Bad Reichenhall ist eine crossmedia Werbeagentur, 
die seit 2001 besteht und mittlerweile über 30 Mitarbeiter beschäftigt. 
Die Agentur unterteilt sich in die vier
Abteilungen: Projektmanagement, Kreation- und Design, Web-Technologie \& Design
sowie IT Services. Unter anderem durch die geografische Lage des Unternehmens 
begründet hat die Agentur ihren Schwerpunkt im Bereich Destinationstourismus.
Dementsprechend sind die meisten Softwarelösungen, die die \mh anbietet, darauf ausgelegt.
Zum Angebotsspektrum des Unternehmens gehören neben dem \ac{CMS} \ct 
auch Softwarelösungen wie digiMELDE als Online-Meldewesen Plattform, makroCARD
für Customer Loyality Management und ein innovatives Werkzeug zur
Erstellung von Gastgeberverzeichnissen -- digiAIR (\Fachbegriff{Web2Print}).


%Das \acs{CMS} \ct wird bereits von vielen großen und kleinen
%Destinationen bei der Pflege derer Webauftritten erfolgreich eingesetzt, wie
% \zB von Berchtesgadener Land Tourismus, Predigtstuhlbahn oder Alpine Welten – Der Bergführer.

\subsection{Projektziel} 
\label{sec:Projektziel}

Das Bestreben des Auftraggebers ist es, durch die neue Website dem Besucher
relevante Informationen zu bieten, Emotionen zu vermitteln, das Erfüllen
moderner Anforderungen, sowie Buchungen und Anfragen zu generieren.

Ein Hauptaugenmerk ist hierbei, eine optimale und geräteunabhängige Darstellung
im Web zu verwirklichen -- \ac{RWD}. Beim \ac{RWD} erfolgt der
grafische Aufbau der Website auf Basis der Anforderung des jeweiligen Endgerätes
auf dem die Website geöffnet wird. Die Umstellung auf \ac{RWD} führt demnach
zur einer vereinfachten und optimierten Bedienung in allen gängigen
Darstellungsgrößen. Daraus erschließt sich das Ziel eine bestmögliche
Benutzererfahrung zu schaffen.

Ein weiteres Anliegen ist die zeitaufwändige und komplizierte Einpflege von
Inhalten und weitere technische Mängel wie \zB dürftige \ac{SEO} mit einem
\ac{CMS} zu umgehen.


\subsection{Projektbegründung} 
\label{sec:Projektbegruendung}
Basierend auf der Kernzielgruppe "`50+"' ist eine übersichtliche und
anpassungsfähige Darstellung von relevanten Informationen signifikant. 
Die statische Darbietung auf verschiedenen Displaygrößen benachteiligt die
\Fachbegriff{Userability} schwerwiegend und führt dazu, dass
überdurchschnittlich viele Nutzer die Website verlassen. Durch das modernere
Erscheinungsbild in Kombination mit \ac{RWD} kann dies verhindert werden.
 
Ebenso unvorteilhaft ist die lückenhafte Suchmaschinenoptimierung, die die Suche
der Website erschwert. Hinzu kommen noch Performance- und
Sicherheitsprobleme, die durch die veraltete Technik der aktuellen Website
ausgelöst werden. Die eben genannten Hindernisse als auch die komplizierte und
zeitaufwändige Einpflege von Inhalten kann durch die Implementierung des
hauseigenen sowie modernen \ac{CMS} \ct umgangen werden.

Auf Grund dieser Lösungsansätze und die dafür nötigen Kompetenzen hat sich die
\kunde entschlossen, der Firma \mh die Entwicklung einer neuen Website in
Auftrag zu geben.