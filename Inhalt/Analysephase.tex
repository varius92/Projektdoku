% !TEX root = ../Projektdokumentation.tex
\section{Analysephase} 
\label{sec:Analysephase}


\subsection{Ist-Analyse} 
\label{sec:IstAnalyse}

Im Verlauf der Ist-Analyse wurde eine Liste mit allen Problemen sowie
Anforderungen und deren technischen Lösungsmöglichkeiten aufgeführt. 

Derzeitig beruht die Website auf einer veralteten \ct (Version 5), die
noch einen komplizierten Ablauf zur Einpflege von Inhalten beherbergt. Dies ist mit
einem hohen Zeitwaufwand sowie einer hohen Fehleranfälligkeit behaftet. Zu dem
fehlen einige technischen Features, wie \zB eine umfangreiche \ac{SEO} und eine
Mediendatenbank zur einfachen Verwaltung von Bilddateien \oae Formaten. 
Außerdem wird die Website momentan auf allen Geräten in der Desktop-Ansicht
präsentiert, was zu einer schlechten Bedienung der Steuerelementen führt.
Möchte sich der Kunde über ein Smartphone informieren, wird er
Schwierigkeiten beim Navigieren der Website haben. 

Der mit \ac{RWD} konzeptionierte Relaunch der Website mit Anbindung der
aktuellen \ct Version (6) löst alle oben benannten technischen Probleme und haltet alle
Kundenanforderungen ein.

\subsection{Wirtschaftlichkeitsanalyse}
\label{sec:Wirtschaftlichkeitsanalyse}
Aufgrund der techschnischen und \Fachbegriff{Usability} Problemen, die in
Kapitel \ref{sec:Projektbegruendung} \nameref{sec:Projektbegruendung}
sowie \ref{sec:IstAnalyse} \nameref{sec:IstAnalyse} beschrieben werden,
ist die Entwicklung in Form eines Relaunches erforderlich. In den folgenden Abschnitten wird die Gerechtfertigung aus wirtschaftlichen Standpunkten näher
erläutert.

\subsubsection{\gqq{Make or Buy}-Entscheidung}
\label{sec:MakeOrBuyEntscheidung}
In diesem Projekt ist eine "`Make or Buy"'-Entscheidung nicht gegeben, da die
Anforderungen des Kunden \kunde in einigen Fällen zu spezifiziert sind.

\subsubsection{Projektkosten}
\label{sec:Projektkosten}
\begin{itemize}
	\item Welche Kosten fallen bei der Umsetzung des Projekts im Detail an (\zB Entwicklung, Einführung/Schulung, Wartung)?
\end{itemize}

\paragraph{Beispielrechnung (verkürzt)}
Die Kosten für die Durchführung des Projekts setzen sich sowohl aus Personal-, als auch aus Ressourcenkosten zusammen.
Laut Tarifvertrag verdient ein Auszubildender im dritten Lehrjahr pro Monat \eur{1000} Brutto. 

\begin{eqnarray}
8 \mbox{ h/Tag} \cdot 220 \mbox{ Tage/Jahr} = 1760 \mbox{ h/Jahr}\\
\eur{1000}\mbox{/Monat} \cdot 13,3 \mbox{ Monate/Jahr} = \eur{13300} \mbox{/Jahr}\\
\frac{\eur{13300} \mbox{/Jahr}}{1760 \mbox{ h/Jahr}} \approx \eur{7,56}\mbox{/h}
\end{eqnarray}

Es ergibt sich also ein Stundenlohn von \eur{7,56}. 
Die Durchführungszeit des Projekts beträgt 70 Stunden. Für die Nutzung von Ressourcen\footnote{Räumlichkeiten, Arbeitsplatzrechner etc.} wird 
ein pauschaler Stundensatz von \eur{15} angenommen. Für die anderen Mitarbeiter wird pauschal ein Stundenlohn von \eur{25} angenommen. 
Eine Aufstellung der Kosten befindet sich in Tabelle~\ref{tab:Kostenaufstellung} und sie betragen insgesamt \eur{2739,20}.
\tabelle{Kostenaufstellung}{tab:Kostenaufstellung}{Kostenaufstellung.tex}


\subsubsection{Amortisationsdauer}
\label{sec:Amortisationsdauer}

Aufgrund dessen, dass es sich um ein externes Projekt handelt, welches sich bei
der Bezahlung der Rechnung für die \mh bereits alle Kosten deckt (inkl. Gewinn), ist
eine Amortisationsdauer für das Unternehmen nicht notwendig.
Auf der Seite des Kunden kann eine Berechnung der Amortisierungszeit nicht exakt
durchgeführt werden, da sich der positive wirtschaftliche Effekt angesichts der
verbesserten \Fachbegriff{Usability} und der Zeiteinsparrung bei der
Inhaltspflege nur sehr schlecht einschätzen lässt.


\subsection{Anwendungsfälle}
\label{sec:Anwendungsfaelle}

Die Übersicht der anfallenden Anwendungsfälle, die durch den Relaunch der
Website abgedeckt werden sollen, finden sich in Form von Use-Case-Diagrammen im
\Anhang{app:UseCase}. Diese bilden die nötigen Funktionen ab, die aus Sicht
des Redakteurs sowie des herkömmlichen Benutzer unerlässlich sind.

\begin{itemize}
	\item Welche Anwendungsfälle soll das Projekt abdecken?
	\item Einer oder mehrere interessante (!) Anwendungsfälle könnten exemplarisch durch ein Aktivitätsdiagramm oder eine \ac{EPK} detailliert beschrieben werden. 
\end{itemize}


\subsection{Qualitätsanforderungen}
\label{sec:Qualitaetsanforderungen}

Da es sich bei \ct um ein fertiges \ac{CMS} handelt, welches alle
Entwicklungsphasen bereits durchlaufen hat, sind in diesem Bezug keine
Qualitätsanforderungen vorhanden. Die im \Fachbegriff{Frontend} nötigen
Anforderungen im Hinblick auf Performance und \Fachbegriff{Usability} werden im
Folgenden erläutert.

Infolge der verschiedenen Darstellungen je nach Browser und
benutzerdefinierten Einstellungen ist eine ständige Überprüfung auf
Darstellugnsfehler in der Implemmentierungsphase unabdingbar.
Um die Gestaltungs- und Layoutelemente reakionsfähiger zu bilden, werden
relative Bezugsgrößen für die Maßangabe verwendet. Diese werden \zB bei
Schriftgrößen, Bildgrößen und Layout-Grids verwendet.Skalierende Icons und 
änhliche Grafiken werden zwecks der scharfen Darstellung in Form von \ac{SVG}
eingebettet.

Um eine angenehme und flüssige Bedienung am Smartphone zu gewährleisten, wird
auf Javascript falls möglich verzichtet.

\subsection{Lastenheft/Fachkonzept}
\label{sec:Lastenheft}
\xx Anhang Lastenheft
\begin{itemize}
	\item Auszüge aus dem Lastenheft/Fachkonzept, wenn es im Rahmen des Projekts erstellt wurde.
	\item Mögliche Inhalte: Funktionen des Programms (Muss/Soll/Wunsch), User Stories, Benutzerrollen
\end{itemize}

\paragraph{Beispiel}
Ein Beispiel für ein Lastenheft findet sich im \Anhang{app:Lastenheft}. 

\Zwischenstand{Analysephase}{Analyse}
