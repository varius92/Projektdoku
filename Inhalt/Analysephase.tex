% !TEX root = ../Projektdokumentation.tex
\section{Analysephase} 
\label{sec:Analysephase}


\subsection{Ist-Analyse} 
\label{sec:IstAnalyse}

Im Verlauf der Ist-Analyse wurde eine Liste mit allen Problemen sowie
Anforderungen und deren technischen Lösungsmöglichkeiten aufgeführt. 

Derzeitig beruht die Website auf einer veralteten Version von \ct (5),
die noch einen komplizierten Ablauf zur Einpflege von Inhalten beherbergt. Dies ist mit
einem hohen Zeitaufwand sowie einer hohen Fehleranfälligkeit behaftet. Zudem
fehlen einige technischen Features, wie \zB umfangreiche \ac{SEO} und eine
Mediendatenbank zur einfachen Verwaltung von Bilddateien \oae Formaten. 
Außerdem wird die Website momentan auf allen Geräten in der Desktop-Ansicht
präsentiert, was zu einer schlechten Bedienung der Steuerelemente führt.
Möchte sich der Kunde über ein Smartphone informieren, wird er
Schwierigkeiten beim Navigieren der Website haben. 

Der mit \ac{RWD} konzeptionierte Relaunch der Website mit Anbindung der
aktuellen \ct Version (6) löst alle oben benannten technischen Probleme und hält
alle Kundenanforderungen ein.

\subsection{Wirtschaftlichkeitsanalyse}
\label{sec:Wirtschaftlichkeitsanalyse}
Aufgrund der Technischer- und \Fachbegriff{Usability}-Problemen, die in
Kapitel \ref{sec:Projektbegruendung} \nameref{sec:Projektbegruendung}
sowie \ref{sec:IstAnalyse} \nameref{sec:IstAnalyse} beschrieben werden,
ist die Entwicklung in Form eines Relaunches erforderlich. In den folgenden Abschnitten wird die Gerechtfertigung aus wirtschaftlichen Standpunkten näher
erläutert.

\subsubsection{\gqq{Make or Buy}-Entscheidung}
\label{sec:MakeOrBuyEntscheidung}
In diesem Projekt ist eine "`Make or Buy"'-Entscheidung nicht gegeben, da die
Anforderungen des Kunden \kunde in einigen Fällen zu spezifiziert sind.

\subsubsection{Projektkosten}
\label{sec:Projektkosten}

In diesem Projekt setzen sich die Kosten für die Durchführung des Projekts
nur aus Personalkosten zusammen. Zur besseren Überischt für den Kunden erfolgte
die Abrechnung über einen Mischstundensatz von 65€. Dieser Satz ergibt sich
durch den Mittelwert der Stundensätze für Projektmanagement, Design und Entwicklung.

Die Aufstellung der Kosten kann in der Tabelle~\ref{tab:Kostenaufstellung}
entnommen werden.
\tabelle{Kostenaufstellung}{tab:Kostenaufstellung}{Kostenaufstellung.tex}



\subsubsection{Amortisationsdauer}
\label{sec:Amortisationsdauer}

Aufgrund dessen, dass es sich um ein externes Projekt handelt, welches sich bei
der Bezahlung der Rechnung für die \mh bereits alle Kosten deckt (inkl. Gewinn), ist
eine Amortisationsdauer für das Unternehmen nicht notwendig.
Auf der Seite des Kunden kann eine Berechnung der Amortisierungszeit nicht exakt
durchgeführt werden, da sich der positive wirtschaftliche Effekt angesichts der
verbesserten \Fachbegriff{Usability} und der Zeiteinsparung bei der
Inhaltspflege nur sehr schlecht einschätzen lässt.


\subsection{Anwendungsfälle}
\label{sec:Anwendungsfaelle}

Die Übersicht der anfallenden Anwendungsfälle, die durch den Relaunch der
Website abgedeckt werden sollen, finden sich in Form eines Use-Case-Diagramms
im \Anhang{app:UseCase}. Es bildet die nötigen Funktionen ab, die aus Sicht
des Redakteurs \bzw Administrators sowie des herkömmlichen Benutzer unerlässlich
sind. Die zwei Aktionen "`Aufruf der Konfiguration und Rechteverwaltung"' sowie
"`Einpflege von Inhalten"' sind zusammengefasste Punkte, die theoretisch selber
Use-Case-Diagramme beherbergen.

\subsection{Qualitätsanforderungen}
\label{sec:Qualitaetsanforderungen}

Da es sich bei \ct um ein fertiges \ac{CMS} handelt, welches alle
Entwicklungsphasen bereits durchlaufen hat, sind in diesem Bezug keine
Qualitätsanforderungen vorhanden. Die im \Fachbegriff{Frontend} nötigen
Anforderungen im Hinblick auf Performance und \Fachbegriff{Usability} werden im
Folgenden erläutert.

Infolge der verschiedenen Darstellungen je nach Browser und
benutzerdefinierten Einstellungen ist eine ständige Überprüfung auf
Darstellungsfehler in der Implementierungsphase unabdingbar.
Um die Gestaltungs- und Layoutelemente reaktionsfähiger zu bilden, werden
relative Bezugsgrößen für die Maßangabe verwendet. Diese werden \zB bei
Schriftgrößen, Bildgrößen und Layout-Grids verwendet. Skalierende Icons und 
änhliche Grafiken werden zwecks der scharfen Darstellung in Form von \ac{SVG}
eingebettet.

Um eine angenehme und flüssige Bedienung am Smartphone zu gewährleisten, wird
auf Javascript falls möglich verzichtet.

\subsection{Lastenheft}
\label{sec:Lastenheft}

Die Anforderungen im Lastenheft wurden im Laufe eines ausführlichen Gespräches
mit dem Kunden vom Projektmanagement im Form eines Grobkonzeptes aufgeführt. Ein
Auszug dafür befindet sich im Anhang \ref{app:Lastenheft} auf Seite
\pageref{app:Lastenheft}.

\clearpage
