% !TEX root = ../Projektdokumentation.tex
\section{Projektplanung} 
\label{sec:Projektplanung}


\subsection{Projektphasen}
\label{sec:Projektphasen}

Für die Umsetzung des Projektes standen 70 Stunden zur Verfügung. Diese wurden vor
Beginn des Projektes auf verschiedene Phasen der Softwarentwicklung verteilt.
Die Folgende Tabelle~\ref{tab:Zeitplanung} zeigt eine grobe zeitmäßige Planung
der Entwicklungszeit.
Eine detailliertere Zeitplanung kann im \Anhang{app:Zeitplanung} entnommen werden.
\tabelle{Grobe Zeitplanung}{tab:Zeitplanung}{ZeitplanungKurz}




\subsection{Abweichungen vom Projektantrag}
\label{sec:AbweichungenProjektantrag}

\begin{itemize}
	\item Sollte es Abweichungen zum Projektantrag geben (\zB Zeitplanung, Inhalt des Projekts, neue Anforderungen), müssen diese explizit aufgeführt und begründet werden.
\end{itemize}


\subsection{Ressourcenplanung}
\label{sec:Ressourcenplanung}
Alle im Projekt verwendeten Ressourcen werden im \Anhang{list:Ressourcenplanung}
aufgeführt. Diese werden in Soft- und Hardwareressourcen, als auch in Personal
unterteilt. Es wurden bereits vorhandene Ressourcen der \mh verwendet, um die
Kosten gering zu halten. Ansonsten wurde ausschließlich lizenzfreie Software (wie \zB
\FachBegriff{Open Source}) für das Projekt herangezogen.



\subsection{Entwicklungsprozess}
\label{sec:Entwicklungsprozess}
AGIL?
http://blog.avenit.de/beitrag/2015/01/16/workflow-und-konzeptionsphase-im-responsive-webdesign/
https://de.wikipedia.org/wiki/Agile_Softwareentwicklung
\begin{itemize}
	\item Welcher Entwicklungsprozess wird bei der Bearbeitung des Projekts verfolgt (\zB Wasserfall, agiler Prozess)?
\end{itemize}
