% !TEX root = ../Projektdokumentation.tex
\section{Projektplanung} 
\label{sec:Projektplanung}


\subsection{Projektphasen}
\label{sec:Projektphasen}

Für die Umsetzung des Projektes standen 70 Stunden zur Verfügung. Diese wurden vor
Beginn des Projektes auf verschiedene Phasen der Softwarentwicklung verteilt.
Die Folgende Tabelle~\ref{tab:Zeitplanung} zeigt eine grobe zeitmäßige Planung
der Entwicklungszeit.
Eine detailliertere Zeitplanung kann im \Anhang{app:Zeitplanung} entnommen werden.
\tabelle{Grobe Zeitplanung}{tab:Zeitplanung}{ZeitplanungKurz}




\subsection{Abweichungen vom Projektantrag}
\label{sec:AbweichungenProjektantrag}

\begin{itemize}
	\item Sollte es Abweichungen zum Projektantrag geben (\zB Zeitplanung, Inhalt des Projekts, neue Anforderungen),
	 müssen diese explizit aufgeführt und begründet werden.
\end{itemize}


\subsection{Ressourcenplanung}
\label{sec:Ressourcenplanung}
Alle im Projekt verwendeten Ressourcen werden im \Anhang{list:Ressourcenplanung}
aufgeführt. Diese werden in Soft- und Hardwareressourcen, als auch in Personal
unterteilt. Es wurden bereits vorhandene Ressourcen der \mh verwendet, um die
Kosten gering zu halten. Ansonsten wurde ausschließlich lizenzfreie Software (wie \zB
\FachBegriff{Open Source}) für das Projekt herangezogen.



\subsection{Entwicklungsprozess}
\label{sec:Entwicklungsprozess}
Da das Screendesign von einer externen Werbeagentur bereitgestellt wird, ist
eine agiler Entwicklungsprozess vonnöten. Dadurch kann man ohne großen Aufwand
und flexibel auf Änderungen seitens des Screendesigns reagieren.


Agile Softwareentwicklung zeichnet sich durch selbstorganisierende Teams, sowie eine iterative
und inkrementelle Vorgehensweise aus. Es wird versucht, mit geringem
bürokratischem Aufwand und Regeln auszukommen und sich schnell an Veränderungen anzupassen,
ohne dabei das Risiko für Fehler zu erhöhen. \footnote{Vgl. \cite{wiki:Agile_Softwareentwicklung}}


Da das Projektmanagment im ständigen Kontakt zum Kunden als auch zur externen
Werbeagentur steht, kann ein agiler Entwicklungsprozess gewährleistet werden.
